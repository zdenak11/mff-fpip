\begin{question}[name={Metoda Monte Carlo, metoda molekulární dynamiky},authors={Zdeněk Turek}]
	
	\subsection{Metoda Monte Carlo}
	
	Metoda Monte Carlo spočívá ve statistickém zpracování sady náhodných čísel podle připraveného modelu. Základní myšlenku celé metody lze shrnout do několika kroků:
	\begin{enumerate}
		\item Generování náhodných čísel $\xi_i$
		\item Transformace náhodných čísel $\xi_i \rightarrow \gamma_i$
		\item Dosazení do modelu
		\item Statistické vyhodnocení
	\end{enumerate}
	
	\subsection{Metoda molekulární dynamiky}
	
	Metoda molekulární dynamiky spočívá v počítání trajektorií částic na základě pohybových rovnic s dosazením vnějších sil a vzájemných sil působících mezi částicemi navzájem. Podle způsobu řešení pohybových rovnic rozlišujeme tři hlavní postupy:
	\begin{itemize}
		\item Eulerův algoritmus
		\item Verletův rychlostní algoritmus
		\item Leap-Frog algoritmus
	\end{itemize}
	
\end{question}